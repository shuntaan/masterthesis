\chapter{Discussions}\label{chap:discussions}
\begin{chapabstract}

Intro intro intro intro.

\end{chapabstract}

\section{Comparing the calibration with previous results}

Here, we compare our results in Section~\ref{subsec:GammaDistribution} with previous research.
From our samples, we obtain the mean $\gamma=-0.63 \pm 0.07$ with MWA frequencies and $1500\MHz$.
\citet{CalistroRivera2017a} and \citet{Chyzy2018} obtained $-0.78 \pm 0.24$ and $-0.56 \pm 0.11$, respectively.
The difference of these might result from the selection of galaxy samples.
While we use nearby galaxies within $25\,\mr{Mpc}$, \citet{CalistroRivera2017a} do 758 galaxies up to $z\sim2$, and \citet{Chyzy2018} do 118 galaxies up to $z=0.04$.

\citet{Chyzy2018} indicate that the slope is steeper at higher frequencies ($1.3 \sim 5\GHz$), and this might steepen the slope of galaxy samples in \citet{CalistroRivera2017a} which adopt galaxies up to $z\sim2$.
Considering the value and scatter, our result using the Herschel reference sample would be consistent with previous findings.
Indeed, the frequency dependence of the low-frequency emission and its relation with IR emission is still uncertain.



\section{Radio SFR uncertainty}

In Section~\ref{subsec:sfrfromlowradio}, we show the consistency of our SFR calibrations using the low-frequency emission.
For calculating the more accurate radio SFR, we need its spectral energy distribution for each galaxy, as we have already shown in Section~\ref{subsec:sfrfromlowradio}.
Since star-forming galaxies have a wide variety of $\qn$ at low frequencies ($\sim 0.53\,\mr{dex}$ in Figure~14, 15; \citealt{CalistroRivera2017a}), and still we do not know the physical details, the radio SFR calibration has considerable uncertainty.
Substituting averaged $\gamma$ and $\q{1500\MHz}$ obtained from our sample galaxies into Equation~\ref{eq:sfrfromradio} yields the following equation:

\begin{equation}\label{eq:sfrradio_calibration}
    \mr{SFR}_{\mr{Radio},\,150\MHz} = \brp{1.03\pm0.27} \times 10^{-29} \times \brp{\frac{L_{\mr{Radio},\,150\MHz}}{\mr{erg}\,\mr{s}^{-1}}}
\end{equation}
where we substitute the median values of $\gamma=-0.63$, $\q{1500\MHz}=2.48$ and $\nu=150\MHz$ into Equation~\ref{eq:sfrfromradio}.

\citet{CalistroRivera2017a} have obtained the coefficient of $0.76 \pm 0.08$ at $z = 0$ (Equation~11) from their calibration.

We should keep our mind that the SFR calibration has non-negligible uncertainty, possibly caused by the galaxy selection and the variety of $\qn$ at low frequencies.
We need the radio spectral with multi-band observation for estimating SFR accurately.



%\bibliographystyle{mnras}
%%\bibliography{example} % if your bibtex file is called example.bib
%\bibliography{masterthesis}
