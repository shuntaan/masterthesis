\chapter{Summary and Future prospects}\label{chap:summary}
%\begin{chapabstract}
%
%Intro intro intro intro.
%
%\end{chapabstract}

In this study, we investigate 18 star-forming galaxies in the HRS catalog for their global relation of the low-frequency radio with IR luminosities.
We find a single power-law assumption for the frequency dependence of $\qn$ is valid across MWA frequencies and $1.5\GHz$, and their slope $\gamma$ is consistent with previous studies.
We also investigate the consistency of the radio SFR expected to be an extinction-free indicator with SFR calculated from other indicators.
In this study, we calculated the radio SFR in two ways.
The first one is to use individual $\q{1500\MHz}$ and $\gamma$ for calculating SFR.
In this case, the radio SFR is consistent with $\mr{SFR}\msb{IR}$ within $20\%$ error.
Another one is to use the averaged $\q{1500\MHz}$ and $\gamma$ and derive the calibration applied for all galaxies.
This calibration gives us one calibration equation and we can calculate SFR even we have only a single band luminosity.
However, this calibration has a two times more significant error than the former one although it is consistent with $\mr{SFR}\msb{IR}$.

These results suggest that the spectral information for each galaxy is needed to estimate its SFR accurately because of a wide variety of $\qn$.
After all, we propose that future observation with several radio bands is required at low frequencies for measuring accurate SFRs.

For further understanding of the low-frequency properties in star-forming galaxies, we need more samples from multi-band observation.
The updated GLEAM survey will reveal physical details with an order of magnitude better angular-resolution and sensitivity than the latest survey.
