\chapter{Summary and Future Prospects}\label{chap:summary}
%\begin{chapabstract}
%
%Intro intro intro intro.
%
%\end{chapabstract}

In this study, we investigated the frequency dependence of the global low-frequency radio and IR luminosity relation, as measures of SF activity in nearby star-forming galaxies.
For this study, we selected star-forming galaxies and their radio data from the HRS catalog \citep{Boselli2010} and the GLEAM survey catalog \citep{Hurley-Walker2017a}, respectively. We found that 18 star-forming galaxies in the HRS catalog have a radio counterpart in the GLEAM survey which are reliable samples for our study.

Here, we summarize our results and prospects as follows:

\begin{enumerate}
    \item We found that a single power-law fitting was valid for modeling the relation of radio with IR luminosities from MWA frequencies ($72\mbox{--}231\MHz$) to $1500\MHz$.
        The mean of $\gamma$, which shows the frequency dependence of $\qn$, is $-0.63\pm0.07$.
        The result is consistent with \citet{CalistroRivera2017a} and \citet{Chyzy2018} within the error.
        The difference of the mean value among these results might be attributed to the selection of galaxy samples.
    \item We checked the consistency of the SFR calculated from the low-frequency radio emission was consistent with SFRs obtained from IR luminosities $\brp{\mr{SFR}\msb{IR}}$.
        Our result also showed that the radio SFR had less scatter when we used the fitting result of each galaxy for the calibration than used the averaged quantities from all galaxy samples.
        The former method had two times less scatter than the latter.
    \item We also confirmed that the radio SFR was consistent with the SFR obtained from FUV emission dust corrected by the IR emission at $24\micron$.
        Although this indicator traces the direct stellar radiation from young massive stars and possibly different from the $\mr{SFR}\msb{IR}$, the radio SFR with individual parameters was also consistent with $\mr{SFR}\msb{FUV+24mic}$ within a factor of 2.
        We also found that calculating the radio SFR with the averaged parameters brings overestimation for some galaxies whose $\qn$ is significantly smaller than that of other galaxies.
    \item The galaxy properties, such as the active galactic nuclei and the \nh-deficiency, possibly affect the spectral energy distribution at low frequencies (also $\gamma$) and the SFR estimation.
        However, it is still difficult to distinguish those galaxies from only $\qn$ values at present.
        For the study of high-$z$ galaxies, we should find the way to identify the galaxy type with only radio emissions because of the lack of telescopes for the blind survey.
        We need further investigation into the relation of galaxy properties with radio luminosities in star-forming galaxies.
    \item For further understanding of the low-frequency properties in star-forming galaxies, we need more samples from multi-band observation.
        The updated GLEAM survey will reveal physical details with an order of magnitude better angular-resolution and sensitivity than the latest survey.
\end{enumerate}

%After all, we propose that future observation with several radio bands is required at low frequencies for measuring accurate SFRs.

