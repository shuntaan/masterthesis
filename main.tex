%%%%%%%%%%%%%%%%%%%%%%%%%%%%%%%%%%%%%%%%%%%%%%%%%%%%%%%%%%%%%%%%%%%%%%%%%%
% NOTE: THIS IS A ROUGH TEMPLATE THAT WILL MOST LIKELY NEED CHANGES TO
% CONFORM TO YOUR INDIVIDUAL SITUATION
% THERE IS NO GUARANTEE OF ACCURACY OR COMPLETENESS TO THE TEMPLATE
%%%%%%%%%%%%%%%%%%%%%%%%%%%%%%%%%%%%%%%%%%%%%%%%%%%%%%%%%%%%%%%%%%%%%%%%%%

%\documentclass[12pt,a4paper,twoside,openright,final,titlepage]{report}
\documentclass[12pt,a4paper,oneside,openright,final,titlepage]{report}
\usepackage[utf8]{inputenc}
\usepackage{amsmath}
\usepackage{amsfonts}
\usepackage{amssymb}
\usepackage{makeidx}
\usepackage[dvipdfmx]{graphicx}
\usepackage[dvipdfmx]{color}
\usepackage{layout}
\usepackage{lscape}
\usepackage{xfrac}
\usepackage{tabularx}
\usepackage{rotating}
\usepackage{longtable}
%\usepackage{todonotes}
\usepackage{booktabs}   % Showing the table from pandas
\usepackage{lscape}     % Enable rotating
\usepackage[colorlinks=true,
							citecolor=black,
							linkcolor=black,
							urlcolor=red,
							linktocpage=true,
							hyperfootnotes=false,
                            dvipdfmx]{hyperref}

\usepackage[square,sort,numbers,authoryear]{natbib}
%\usepackage{chapterbib}
\setlength\bibhang{.3in}


\usepackage{lineno}
\usepackage{setspace}
\onehalfspacing%
\usepackage{microtype}
\usepackage{color}
\usepackage{fancyhdr}
\usepackage[labelfont=bf]{caption}
% For electronic submission use:
\usepackage[inner=2cm,outer=2cm,top=2cm=bottom=2cm]{geometry}
% For printing use:
%\usepackage[inner=4cm,outer=2cm,top=2cm=bottom=2cm]{geometry}
\usepackage{tocbibind}


\renewcommand{\chaptermark}[1]{\markboth{\MakeUppercase{\thechapter. #1 }}{}}
\renewcommand{\sectionmark}[1]{}
%%%%% AUTHORS - PLACE YOUR OWN COMMANDS HERE %%%%%

% Please keep new commands to a minimum, and use \newcommand not \def to avoid
% overwriting existing commands. Example:
%\newcommand{\pcm}{\,cm$^{-2}$}	% per cm-squared
\newcommand{\mc}[1]{\mathcal{#1}}
\newcommand{\mr}[1]{\mathrm{#1}}
\newcommand{\msb}[1]{_{\mathrm{#1}}}
\newcommand{\msu}[1]{^{\mathrm{#1}}}
\newcommand{\brp}[1]{\left(#1\right)}
\newcommand{\brb}[1]{\left[#1\right]}
\newcommand{\brc}[1]{\left\{#1\right\}}
\newcommand{\bra}[1]{\left\langle#1\right\rangle}
\newcommand{\arcs}{\,\mathrm{arcsec}}
\newcommand{\arcm}{\,\mathrm{arcmin}}
\newcommand{\GHz}{\,\mathrm{GHz}}
\newcommand{\MHz}{\,\mathrm{MHz}}
\newcommand{\micron}{\,\mathrm{\mu m}}
\newcommand{\nh}{H\textsc{i}}
\newcommand{\ih}{H\textsc{ii}}
\newcommand{\qn}{q_{\nu}}
\newcommand{\q}[1]{q_{\mathrm{#1}}}
%%%%%%%%%%%%%%%%%%%%%%%%%%%%%%%%%%%%%%%%%%%%%%%%%%

\fancyhf{}
\fancyhead[RO]{\bfseries\rightmark}
\fancyhead[LE]{\bfseries\leftmark}
\fancyfoot[C]{\thepage}
\pagestyle{fancy}

\newcommand\frontmatter{\pagenumbering{roman}}
\newcommand\mainmatter{\cleardoublepage\pagenumbering{arabic}}

\newenvironment{chapabstract}{%
	\begin{center}%
		\singlespacing%
		\bfseries ABSTRACT
	\end{center}}%

\begin{document}

%	\layout
	\begin{titlepage}
		\centering
		\vfill
		\textit{This thesis is presented for the master's degree of Nagoya University}\\
        \vspace{5cm}
		{\bfseries\LARGE
				Estimation of galaxy SFRs from low radio frequencies \\}

        \vspace{1.5cm}

		%\includegraphics[width=.40\linewidth]{UWA_new}
        \vspace{6cm}
        Shunraro YOSHIDA (ID:\@261801444)\\
		%B.Sc. in FieldA; B.Sc. in FieldB; M.Sc in FieldC \\
		March 2020\\
        \vspace{0.5cm}
        The laboratory of Galaxy Evolution ($\Omega$ lab.)\\
        Division of Particle and Astrophysical Science, Graduate School of Science\\
        Nagoya University, Japan\\
        \vspace{1.5cm}
 		\textbf{Supervisors:}
 		    Prof.\ Dr.\ Tsutomu\ T.\ Takeuchi\\
 			Dr.\ Barbara Catinella\\
 			Dr.\ Luca Cortese\\
 			Dr.\ O.\ Ivy Wong\\
	\end{titlepage}
	%\maketitle
	\frontmatter
%	\linenumbers
	\clearpage
	\thispagestyle{empty}
	\phantom{a}
	\vfill
	\vfill

%	\section*{Thesis Declaration}
\vskip 0.8cm
\begin{flushleft}
I, insert name, certify that: [DELETE ALL IRRELEVANT STATEMENTS]
\vskip 0.5cm 
This thesis has been substantially accomplished during enrolment in the degree.
\vskip 0.25cm
This thesis does not contain material which has been accepted for the award of any other degree or diploma in my name, in any university or other tertiary institution.
\vskip 0.25cm
No part of this work will, in the future, be used in a submission in my name, for any other degree or diploma in any university or other tertiary institution without the prior approval of The University of Western Australia and where applicable, any partner institution responsible for the joint-award of this degree.
\vskip 0.25cm
This thesis does not contain any material previously published or written by another person, except where due reference has been made in the text. 
\vskip 0.25cm
The work(s) are not in any way a violation or infringement of any copyright, trademark, patent, or other rights whatsoever of any person.
\vskip 0.25cm
The research involving human data reported in this thesis was assessed and approved by The University of Western Australia Human Research Ethics Committee. Approval \#: [insert approval number(s)].
\vskip 0.25cm
Written patient consent has been received and archived for the research involving patient data reported in this thesis.
\vskip 0.25cm
The research involving animal data reported in this thesis was assessed and approved by The University of Western Australia Animal Ethics Committee. Approval \#: [insert approval number(s)].
\vskip 0.25cm
The research involving animals reported in this thesis followed The University of Western Australia and national standards for the care and use of laboratory animals.
\vskip 0.25cm
The following approvals were obtained prior to commencing the relevant work described in this thesis: [List approvals here].
\vskip 0.25cm
Third party editorial assistance was provided in preparation of the thesis by [insert name/company].
\vskip 0.25cm
The work described in this thesis was funded by [insert name of grant and grant identification numbers].
\vskip 0.25cm
Technical assistance was kindly provided by [insert name of assistant(s)] for [insert description of assistance] that is described in [insert location in thesis].
\vskip 0.25cm
This thesis contains published work and/or work prepared for publication, some of which has been co-authored. 
\vskip 0.5cm
Signature: [Sign here]
\vskip 0.25cm
Date: [Insert date here]
\end{flushleft}


	\begin{abstract}
        We present a global relation between the low-frequency and infrared (IR) emissions in star-forming galaxies compiled by the Herschel Reference Survey.
        The GaLactic Extragalactic All-sky MWA (GLEAM) survey operated by the Murchison Widefield Array (MWA) allows us to examine the relation and its frequency dependence with their 20 narrow bands at $72-231\MHz$.
        These examinations are important for ensuring the reliability of the radio SFR\@.
        In this study, we focus on 18 star-forming galaxies whose radio emission is detected by the GLEAM survey.
        These galaxies show that a single power-law fitting is valid for understanding how the relation between the radio and IR luminosities varies across MWA frequencies to $1.5\GHz$.
        We also investigate the consistency of the Star Formation Rate (SFR) calculated from the low-frequency emission with that from other indicators.
        Although this low-frequency emission has an advantage of the extinction-free indicator, the SFR calibration with averaged spectral parameters has a non-negligible uncertainty due to the variety of the radio to IR relation.
        We propose to use the individual spectral energy distribution for calculating the radio SFR for each star-forming galaxies with less uncertainty.
	\end{abstract}

	\chapter*{\Large Acknowledgments}

I would like to thank Prof. Tsutomu T. Takeuchi, Dr. Luca Cortese, Dr. O. Ivy Wong and Dr. Barbara Catinella for their extraordinary support in my master's research.
I would like to thank my colleagues also for their helpful and enjoyable discussion with me.
You always supported me when I was suffering the research and my life.
This project would have been impossible without the funding from Nagoya University (Overseas Challenge Program for Young Researchers 2019).


%    \section*{AUTHORSHIP DECLARATION: CO-AUTHORED PUBLICATIONS}
\begin{flushleft}
This thesis contains work that has been [published and/or prepared for publication]. 
\vskip 0.5cm
\begin{tabular}{l}
		Details of the work: \\
        (insert text)      \\ 
        Location in thesis: \\
        (insert text)      \\ 
        Student contribution to work: \\
        (insert text) \\
		Co-author signatures and dates: \\
        (insert text)
\end{tabular} 
\vskip 0.5cm
\begin{tabular}{l}
		Details of the work: \\
        (insert text)      \\ 
        Location in thesis: \\
        (insert text)      \\ 
        Student contribution to work: \\
        (insert text) \\
		Co-author signatures and dates: \\
        (insert text)
\end{tabular} 
\vskip 0.5cm
\begin{tabular}{l}
Student signature: [insert signature] \\
Date: [insert date]\\
\end{tabular}
\vskip 0.5cm
\begin{tabular}{l}
I, [name of coordinating supervisor] certify that the student statements \\ regarding their contribution to each of the works listed above are correct. \\
Coordinating supervisor signature: [insert signature]\\
Date: [insert date]
\end{tabular}
\end{flushleft}



	\setcounter{tocdepth}{2}
	\tableofcontents
	\listoffigures
	\listoftables
	\mainmatter%
	\raggedbottom%
	\chapter{Introduction}
\begin{chapabstract}

Intro intro intro intro.

\end{chapabstract}
\section{Tables}

Reference Table \ref{tab:Table1}.  And blah blah blah.

\begin{table}[t]
\caption[TOC Table Description]{Caption.}
	\centering
	\begin{tabular}{lcc}
		\hline
        {\textbf{Setting}}                      & \multicolumn{2}{c}{\textbf{Mt C/y}} \\
		                                                           &         Min          &     Max      \\ \hline
		AAA                         &          40          &      66      \\
		BBB                               &          14          &      66      \\
		CCC                                       &          18          &      43      \\
		DDD      &          4           &    $>$12     \\
		EEE                                  &          0           &      47      \\
		FFF                    & 1$ \times $10$^{-4}$ &      52      \\
		GGG &          8           &      42      \\ \hline
	\end{tabular}
    \label{tab:Table1}
\end{table}

\section{Figures}
Reference Figure~\ref{fig:Fig1}.

\subsection{This is a subsection}
\begin{figure}[t]
	\centering
	\includegraphics[width=.6\linewidth]{Chapter_1/Figs/Fig1.pdf}
	\caption[TOC Figure Description]{Caption.}
	\label{fig:Fig1}
\end{figure}
\subsubsection{This is a subsubsection}
Citations are like \cite{goossens93,AbedonHymanThomas2003}.  Or maybe \cite{Abedon1994} said something.  Or \cite{Cerveny} which is an example of how to make a bib file that includes an author whose name begins with a non-English character and \cite{forgues96}: an example of referencing a Ph.D. thesis and yet more non-English characters.




\bibliographystyle{abbrvnat}
\bibliography{Chapter_1/ref1}

    %\chapter{Theoretical Background}\label{chap:theory}
\begin{chapabstract}

    In this chapter, I put physical mechanisms of radio emissions related to our study.
    At low frequencies, we observe free-free (Bremsstrahlung) and synchrotron radiations.
    Both emissions are the continuum emission.
    While the free-free radiation is usually dominant from a galaxy at $30 \sim 200\GHz$, the synchrotron radiation is dominant less than $30\GHz$ (Figure~\ref{fig:Condon1992_figure1}).
\begin{figure}[htbp]
	\centering
	\includegraphics[width=.7\linewidth]{Chapter_2/Figures/Condon1992_Figure1.png}
    \caption[Reprint from \citet{Condon1992a} (Figure~1)]{\label{fig:Condon1992_figure1}
        (Reprint from \citet{Condon1992a}, Figure~1)\\
        This figure shows the spectral energy distribution of M82.
        The dotted line shows the dust thermal emission which is dominant at higher than $200\GHz$.
        The dashed line shows the free-free radiation from the \ih~regions around massive stars,which is dominant at $30 \sim 200\GHz$.
        The dot-dash line shows the synchrotron radiation emitted by the high energy electrons, which is dominant at less than $30\GHz$.
    }
\end{figure}

\end{chapabstract}



\section{Free-free radiation}

The free electrons produce free-free radiation by scattering off ions.
In star-forming galaxies, the radiation source of this emission is the \ih~region where young massive (OB) stars ionized most of the hydrogen atoms.
In this section, we consider the radiation mechanism of free-free radiation.
Here, we consider only electrons emit the radiation because an electron is much more accelerated than an ion due to the difference of their mass (an electron is 1840 times lighter than a proton).

Firstly, I consider the simplest case that a single electron passes by the ion and emits the radiation.
Note that the path of the electron does not change after the interaction because the energy of radio emission is much smaller than the mean electron energy in a plasma.

\begin{equation}\label{eq:essential_radio4n12}
    \frac{E\msb{10\GHz}}{\langle E_e \rangle} = \frac{h \times 10^{10}\,\mr{Hz}}{3kT / 2} = \frac{6.63 \times 10^{-27} \mr{erg\,s} \times 10^{10}\,\mr{Hz}}{1.5 \cdot 1.38 \times 10^{-16}\,\mr{erg\,K^{-1}} \cdot 10^4\,\mr{K}} = 3.3 \times 10^{-5}
\end{equation}

During the interaction, the electron is accelerated by the electric field by the ion.
Then, we can write the equation of motion for parallel and perpendicular to the electron's path (Figure~\ref{fig:nrao_radio4n2}):

\begin{figure}[htbp]
	\centering
	\includegraphics[width=.5\linewidth]{Chapter_2/Figures/NRAO_radio4n2.png}
    \caption[The schematic image of the interaction of an electron with the ion]{\label{fig:nrao_radio4n2}
    }
\end{figure}

\begin{align}
    F_{\|} &= m_{\mr{e}} \dot{v}_{\|}=\frac{-Z e^{2}}{r^{2}} \sin \psi=\frac{-Z e^{2} \sin \psi \cos ^{2} \psi}{b^{2}}\label{eq:essential_radio4n15}\\
    F_{\perp} &= m_{\mr{e}} \dot{v}_{\perp}=\frac{Z e^{2}}{r^{2}} \cos \psi=\frac{Z e^{2} \cos ^{3} \psi}{b^{2}}\label{eq:essential_radio4n16}
\end{align}
where $b$ is the impact parameter and $\cos\psi = \frac{b}{r}$.

These accelerations show different shapes of the pulse (Figure\ref{fig:nrao_radio4n3}).
Since the parallel acceleration produces some infrared radiation with the angular frequency $\omega \sim \tau^{-1} = \frac{v}{b}$ ($tau$ is a collision time), its contribution is negligible at radio frequency.

\begin{figure}[htbp]
	\centering
	\includegraphics[width=.7\linewidth]{Chapter_2/Figures/NRAO_radio4n3.png}
    \caption[The acceleration of an electron by an ion]{\label{fig:nrao_radio4n3}
    }
\end{figure}

Therefore, the power of free-free radiation from the acceleration of the electron perpendicular to its velocity is:

\begin{equation}\label{eq:essential_radio4n17}
    P=\frac{2}{3} \frac{e^{2} \dot{v}_{\perp}^{2}}{c^{3}}=\frac{2 e^{2}}{3 c^{3}} \frac{Z^{2} e^{4}}{m_{\mr{e}}^{2}}\brp{\frac{\cos ^{3} \psi}{b^{2}}}^{2}
\end{equation}
where we insert $\dot{v}_{\perp}$ into the Larmor's formula $\brp{P = \frac{2}{3}\frac{q^2\dot{v}^2}{c^3},\,q\,\mr{is\,a\,charge}}$, which shows the power emitted by the accelerated particles.

We can get the total energy of $W$ by the pulse as follows:

\begin{equation}\label{eq:essential_radio4n18}
    W = \int^{\infty}_{-\infty} P \mr{d}t
\end{equation}

As I noted above, the electrons' velocity is constant so that we can change of variables:

\begin{equation}\label{eq:essential_radio4n19}
    v = \frac{\rd x}{\rd t}\ \ \ \mr{and}\ \ \ \tan\psi = \frac{x}{b}
\end{equation}

then,

\begin{equation}\label{eq:essential_radio4n20}
    v=\frac{b\,\rd\brp{\tan \psi}}{\rd t}=\frac{b\,\sec ^{2} \psi\,\rd \psi}{\rd t}=\frac{b\,\rd \psi}{\cos ^{2}\psi\,\rd t}
\end{equation}

and

\begin{equation}\label{eq:essential_radio4n21}
    \rd t=\frac{b}{v} \frac{\rd \psi}{\cos ^{2} \psi}
\end{equation}

Substituting Equation~\ref{eq:essential_radio4n17} and~\ref{eq:essential_radio4n21} into Equation~\ref{eq:essential_radio4n18} yields

\begin{equation}\label{eq:essential_radio4n22}
    W=\frac{2}{3} \frac{Z^{2} e^{6}}{c^{3} m_{\mathrm{e}}^{2} b^{4}} \int_{-\pi / 2}^{\pi / 2} \frac{b}{v} \frac{\cos ^{6} \psi}{\cos ^{2} \psi} \rd \psi=\frac{4}{3} \frac{Z^{2} e^{6}}{c^{3} m_{\mathrm{e}}^{2} b^{3} v} \int_{0}^{\pi / 2} \cos ^{4} \psi\ \rd \psi = \frac{\pi Z^2 e^6}{4 c^3 m_{\mr{e}}^2}\brp{\frac{1}{b^3v}}
\end{equation}
where $\int_{0}^{\pi / 2} \cos ^{4} \psi\,\rd \psi=\frac{3 \pi}{16}$.

The pulse energy $W$ shows the total energy emitted by a single electron interaction characterized by the impact parameter $b$ and the electron's velocity $v$.\\ \vspace{0.2cm}

Secondly, I consider the strength and spectral of free-free radiation from \ih~region with several simple assumptions.



\section{Synchrotron radiation}

Synchrotron radiation is a continuum emission which is usually dominant at less than $30\GHz$ in star-forming galaxies.
The interaction of high energy electrons accelerated by the supernova remnant with the galactic magnetic field causes the radiation.
Since high energy electrons have a power-law energy distribution, we call the radiation ``non-thermal synchrotron radiation''.
If electrons have a much smaller velocity than the light, they emit the cyclotron radiation with the cyclotron frequency $\omega = \frac{eB}{m}$ ($e$ is a charge, $B$ is the strength of the magnetic field, $m$ is a mass of an electron).

In this section, I show the mechanism to emit synchrotron radiation.
Firstly, I focus on synchrotron radiation emitted by a single electron.
In the magnetic field with the strength $B$, an electron has a cyclotron frequency $\omega_g = \frac{eB}{m}$

	\chapter{Data}\label{chap:data}
\begin{chapabstract}

In this chapter, I describe the dataset used for our study.
In Section~\ref{sec:HerschelReferenceSurvey}, I introduce the Herschel Reference Survey Catalog \citep{Boselli2010}.
Here, I explain how they choose galaxies for the catalog and previous studies for them.
In Section~\ref{sec:gleamsurvey}, I introduce the GaLactic Extragalactic All-sky MWA survey, which we obtained the radio data for galaxy samples.

\end{chapabstract}

\section{Herschel Reference Survey (HRS) catalog}\label{sec:HerschelReferenceSurvey}
In this section, I introduce the Herschel Reference Survey (HRS) catalog \citep{Boselli2010} from which we selected galaxy samples.
This survey is one of the Herschel guaranteed time key projects, and originally it was compiled for understanding dust properties and the interstellar medium in nearby galaxies.
The catalog is publicly available and contains 322 galaxies selected with three criteria as follows:

\begin{enumerate}
    \item Volume-limited:\\
        They choose galaxies whose distance from the earth is between $15$ and $25\,\mr{Mpc}$.
        This limitation reduces the distance uncertainty due to the galaxy peculiar motions and the selection effect due to the high-$z$ galaxies.
        The lower limit ($15\,\mr{Mpc}$) also helps us to observe sources within reasonable exposure time because very close galaxies us are extended, and we need too much time for the observation.
    \item $K$-band selection:\\
        They choose galaxies whose 2MASS $K$-band total magnitudes are brighter than $12\,\mr{mag}$ for star-forming and peculiar galaxies (Sa-Sd-Im-BCD), and $8.7\,\mr{mag}$ for quiescent galaxies (E, S0, S0a).
        If there are galaxies whose $K$-band magnitude darker than those values, their measurements are not regarded as accurate photometry because of not enough exposure time.
        The reason why they have selected quiescent galaxies with the more stringent $K$-band selection criteria is these galaxies are expected to have low dust contents, and it is difficult to detect within the reasonable exposure time.
    \item High galactic latitude:\\
        They choose galaxies whose galactic latitude is high enough to minimize the contamination from the galactic center ($b > +55^{\circ}$).
        Also, they select galaxies with the low galactic extinction ($A\msb{B} < 0.2$; \citealt{Schlegel1998}).
\end{enumerate}

The selected galaxies locate in the sky region between $10\msu{h}17\msu{m}< \mr{R.A.}(2000) < 14\msu{h}43\msu{m}$ and $-6^{\circ} < \mr{decl.} < 60^{\circ}$ (Figure~\ref{fig:Boselli2010_figure1}).
HRS galaxies span a wide range of the galaxy density environment from the center of the Virgo cluster to the isolated region.
As a definition, we can regard the HRS sample as an ideal one for studying the galaxy environment.

\begin{figure}[htbp]
	\centering
	\includegraphics[width=.7\linewidth]{Chapter_3/Figures/Boselli2010_Figure1.png}
    \caption[Reprint from Boselli et al. 2010 (Figure~1)]{\label{fig:Boselli2010_figure1}
        (Reprint from Boselli et al. 2010, Figure~1)\\
        This figure shows the sky distribution of the HRS galaxy sample.
        They show the early-type galaxies (E, S0, S0a) and late-type galaxies with circles and crosses, respectively.
        Dashed circles represents the different cloud regions. Each name of the cloud is shown close to each region.
        The red and dark green markers are Virgo galaxies (red: Virgo center, dark green: its outskirts).
    }
\end{figure}

In addition to a wide range of the environment, HRS galaxies distribute a wide range of galaxy morphology (Figure~\ref{fig:Boselli2010_figure2}).

\begin{figure}[htbp]
	\centering
	\includegraphics[width=.8\linewidth]{Chapter_3/Figures/Boselli2010_Figure2.png}
    \caption[Reprint from Boselli et al. 2010 (Figure~2)]{\label{fig:Boselli2010_figure2}
        (Reprint from Boselli et al. 2010, Figure~2)\\
        This figure shows the distribution in the morphology-type of HRS galaxies.
        The shaded histogram represents the distribution in it of only the cluster sample.
        Here, the cluster sample composed of HRS galaxies located in the Virgo A and B clouds.
    }
\end{figure}

Since HRS galaxies are supposed to be well-represented for the whole galaxy population located in the local universe, investigating their physical properties is crucial to understand them.
After \citet{Boselli2010} published the HRS catalog, many studies investigating the physical properties for HRS galaxies have been done until now.
Here, I introduce some of the studies for the HRS sample.
\citet{Cortese2012} investigated their UV and optical properties using the Galaxy Evolution Explorer ({\it GALEX\/};~\citealt{Martin2005}) and SDSS-DR7 \citep{Abazajian2009}.
\citet{Boselli2014} studied their cold gas properties with $^{12}\mr{CO}\brp{1\mbox{--}0}$ observed by the Kitt Peak 12m radio telescope and obtained from the literature data.
They also investigate the \nh~gas obtained from The Arecibo Legacy Fast ALFA (ALFALFA;~\citealt{Giovanelli2005, Haynes2011}) survey.
\citet{Ciesla2014} executed the SED fitting for HRS galaxies with Code Investigating GALaxy Emission (CIGALE;~\citealt{Noll2009}).

Thanks to all of the previous research about the HRS sample, they are well-studied from the X-ray to the radio emission at $1.5\GHz$.
However, the low-frequency around $100\MHz$ is not examined so far.
Since we would like to extend the wavelength range of the HRS sample to around $100\MHz$, in this study, we focus on a subsample of HRS galaxies whose counterpart is detected by the latest low-frequency survey (Section~\ref{sec:gleamsurvey}).



\section{The GaLactic Extragalactic All-sky MWA (GLEAM) survey catalog}\label{sec:gleamsurvey}

In this section, I introduce the GaLactic Extragalactic All-sky MWA (GLEAM) survey \citep{Hurley-Walker2017a}, which we obtained the radio continuum data from.
This survey was operated by the Murchison Widefield Array (MWA) telescope in Western Australia \citep{Tingay2013a}.
It observed a whole southern sky and a northern sky up to $+30^{\circ}$ ($\sim$25,000 $\mathrm{\deg}^2$;~Figure~\ref{fig:HurleyWalker2017_figure11}).
The catalog from this survey is publicly available and contains 307,455 detected radio sources with fluxes at 20 narrow bands between $72$ and $231\MHz$ (each band has $7.68\MHz$ bandwidth).
The observations of this survey were operated by the grouped five bands with $30.72\MHz$ bandwidth.
These grouped bands have the center on $87.7$, $118.4$, $154.2$, $185.0$, $215.7\MHz$, respectively.
Because of the orbcomm satellite interference, They did not observe at $134\mbox{--}139\MHz$.
The observation was executed sequentially as $112\,\mr{s}$ snapshots (each frequency was observed every $10\,\mr{minutes}$).
The sensitivity and angular resolution at $200\MHz$ are $\sim 7\,\mr{mJy}$ and $\sim 2\,\mr{arcmin}$ respectively.
The completeness of this survey at $200\MHz$ is $90\%$ at $\sim 170\mr{mJy}$.
Since this survey allows us to examine the low-frequency spectral energy distribution accurately with its 20 narrow bands, we adopt the radio source catalog for our study.

\begin{figure}[htbp]
	\centering
	\includegraphics[width=.7\linewidth]{Chapter_3/Figures/HurleyWalker_Figure11.png}
    \caption[Reprint from Hurley-Walker et al. 2017 (Figure~11)]{\label{fig:HurleyWalker2017_figure11}
        (Reprint from Hurley-Walker et al. 2017, Figure~11)\\
        This figure shows the observed area by the GLEAM survey (green shaded region).
        \citet{Hurley-Walker2017a} exclude several regions intentionally to minimize the contamination:
        Galactic plane (Absolute Galactic latitude $<10^{\circ}$),
        Ionospherically distorted ($0^{\circ} < \mr{Dec} < +30^{\circ}\ \mr{and}\ 22\msu{h} < \mr{R.A.} < 0\msu{h}$),
        Centaurus A ($13\msu{h}25\msu{m}28\msu{s}\ -43^{\circ}01'09'',\,r=9^{\circ}$),
        Sidelobe reflection of Cen A ($20^{\circ} < \mr{Dec} < +30^{\circ}\ \mr{and}\ 13\msu{h}07\msu{m} < \mr{R.A.} < 13\msu{h}53\msu{m}$),
        Large Magellanic Cloud ($05\msu{h}23\msu{m}35\msu{s}\ -69^{\circ}45'22'',\,r=5.5^{\circ}$) and Small Magellanic Cloud ($00\msu{h}52\msu{m}38\msu{s}\ -72^{\circ}48'01'',\,r=2.5^{\circ}$).
    }
\end{figure}






%\bibliographystyle{mnras}
%%\bibliography{example} % if your bibtex file is called example.bib
%\bibliography{masterthesis}

	\chapter{Methods}
\begin{chapabstract}

Intro intro intro intro.

\end{chapabstract}

\section{Cross Matching}\label{sec:crossmatching}
Although HRS galaxies have been studied in multi-wavelength observations, their spectral energy distribution around $100\MHz$ is not well-understood, where the contribution from synchrotron radiation is much more significant than from free-free emission \citep{Condon1992a}.
Here, I provide a procedure to cross-matching with two different catalogs we have mentioned in previous sections.
Cross-matching is the method widely used in astronomy to obtain additional information from other surveys or catalogs by matching coordinates of each galaxy or blob source within a specified error range.
For executing this method for HRS galaxies and the GLEAM survey catalog, we use Tool for OPerations on Catalogues And Tables (TOPCAT\@; \citealt{Taylor2009}).
TOPCAT is a convenient tool for dealing with catalogs and tables, and it allows us to do the cross-matching easily, even more than two catalogs.

Initially, we assume a $10\,\mr{arcsec}$ error radius for the cross-matching since it is equivalent to the value of $95\%$ error for the astrometry in the GLEAM survey (Section 4.5.5 in \citealt{Hurley-Walker2017a}).
This matching results in a total of 18 galaxies which are identified to have a radio counterpart.
To assess the matching, we compare a separation of counterparts from the center of galaxies with coordinate uncertainties in the GLEAM catalog.
For these galaxies, we find 15 of them have a separation within a 95\% error radius, and others do not.
Although three of them have a larger separation compared to the error radius, we conclude that the matching for all 18 galaxies is correct by the checking of galaxy images (Appendix~\ref{sec:galaxyimages}).
In this paper, we regard a radio blob as a counterpart if the brightest part of each blob where the inside of the contour nearest to the center, is surrounded by the D25 radius.

Next, we extend the error radius up to 120 [arcsec], which is corresponding to the angular resolution of the GLEAM survey at $200\MHz$.
This is because the radio sources are blurred due to the angular resolution of the GLEAM survey and the location of the radio source might be shifted.
We know $120\,\mr{arcsec}$ error radius is quite big for the matching, but this trial gives us the inspiration for the cross-matching with blurred radio sources in the future study.
The cross-matching with the error of $120\,\mr{arcsec}$ suggests that there are 25 new galaxies have a potential counterpart.
To assess these matching, we look at galaxy images one by one (All galaxy images are in Appendix~\ref{sec:galaxyimages}).
With the same condition mentioned above, we identify 21 matches for these galaxies.

Although we have identified a total of 39 matches in the same way, there are six suspicious matches because of interacting counterparts (HRS4, 216, 244, 284) and quite large separations (HRS200, 295) (Figure~\ref{fig:galaxyimages_suspicious}).
For these galaxies, we flag them as suspicious matches, and we do not use them for further analysis.
The distribution of separations for galaxies showed in Figure~\ref{fig:separation}.

\begin{figure}[htbp]
	\centering
	\includegraphics[width=.6\linewidth]{Chapter_4/Figures/Method_separation.pdf}
    \caption[Separation from the cross-matching]{\label{fig:separation}
        This figure shows the distribution of separations.
        Here we put 43 galaxies and color sorted based on the result.
        The blue bar shows galaxies identified to have a radio counterpart, red one does galaxies determined not to have a counterpart, and gray for the suspicious galaxies.
        Most of the matched samples are distributed within a $40\,\mr{arcsec}$ error radius.
    }
\end{figure}

We summarize as the table for 39 galaxies identified to have a radio counterpart in Appendix~\ref{sec:sampletable} and put all galaxy images cross-matched within $120\,\mr{arcsec}$ in Appendix~\ref{sec:galaxyimages}.



\section{Reduce galaxy samples for the secure analysis}\label{sec:reducegalaxysamples}

In this section, we show some steps for selecting galaxies to do a secure analysis.
Since we focus on the relation of galaxy radio emission with star formation activities, we should clarify the radio source and be sure that they are not originally from other sources rather than the star formation.
The synchrotron radiation arisen from the star formation activities should be proportional to the SFR in a galaxy.
Therefore, the radio emission from elliptical galaxies, which are thought to have no more star formation, would trace other radio sources unrelated to the star formation.
These radio sources are considered as Active Galactic Nuclei (AGN), which emits strong radio emission due to the baryon accretion into the supermassive black hole at the center of galaxies irrelevant to the star formation.
According to the morphology of galaxies \citep{Cortese2012}, we identify four elliptical galaxies (HRS49, 138, 178, 241) and not use for further calculation and discussion.
%With these operations, we finally find that there are 37 HRS galaxies are available for further analysis.

After this galaxy selection, we obtain 29 galaxies with a reliable radio counterpart arisen from the star formation activity.

As a next step, we evaluate the signal to noise ratio (SNR) of radio emission at each MWA band.
For reducing the uncertainty caused by observation errors, we assess the peak flux at each narrow band by comparing it to the local noise level and adopt flux values whose SNR is higher than five.
This analysis results in a total of 11 galaxies have no radio fluxes whose SNR is higher than the criterion.
The reason why these radio sources are detected, although they do not have any fluxes with higher SNR, is \citet{Hurley-Walker2017a} determine the detection based on the SNR in the stacking images ($170-231\MHz$).
%\vspace{0.3cm}\\

Finally, after the cross-matching and these procedures, we confirm 18 HRS galaxies are the samples available for further analysis.



\section{Calculating the $q$ parameter}\label{subsec:calculatingq}
In this section, we introduce the method to calculate the $q$ parameter for each galaxy.
The definition of this parameter here is given on the following equation \citep[e.g.][]{Helou1985, Bell2003, CalistroRivera2017a}:

\begin{equation}\label{eq:q_def}
    q_{\nu} \equiv \log\brp{\frac{L\msb{8-1000\,\mu m}\ /\ 3.75 \times 10^{12}}{{\rm erg\,s^{-1}\,Hz^{-1}}}} - \log\brp{\frac{L_{\mr{Radio},\,\nu}}{{\rm erg\,s^{-1}\,Hz^{-1}}}}
\end{equation}

where $L\msb{8-1000\,\mu m}$ is the total rest-frame infrared luminosity among $8 - 1000\,\mr{\mu m}$, which reflects the total dust luminosity, and $3.75\times10^{12}$ is equivalent to the frequency of $80\,\mr{\mu m}$ for correcting the dimension.

Although \citet{Ciesla2014} have already derived the total IR luminosity for most HRS galaxies using the SED fitting method, one of our galaxy samples (HRS163) does not have the value because of the lack of the reliable mid IR flux from the Spitzer telescope.
For consistency, we adopt the total IR luminosity calculated from the same method for all galaxy samples.
To calculate the total IR luminosity, we refer to \citet{Galametz2013}, which derived the calibration relation between combining monochromatic IR luminosity and the total IR dust luminosity.
We show the procedure to calculate total IR luminosity in the following section.



\subsection{Total IR luminosity}\label{subsec:tirluminosity}
In this section, we show how to calculate the total IR luminosity.
For the method of calculating total IR luminosity, we refer to \citet{Galametz2013}, which shows the empirical relations to estimate TIR from Spizer bands ($24, 70\,\mr{\mu m}$), and the Herschel band ($100, 160, 250\,\mr{\mu m}$).
HRS galaxies have the flux data from the Multiband Imaging Photometry for Spitzer (MIPS;~\citealt{Rieke2004, Bendo2012}), the Herschel/PACS \citep{Cortese2014c} and the Herschel/SPIRE \citep{Ciesla2012a}.

\citet{Galametz2013} derived the calibration equation as follows:

\begin{equation}\label{eq:galametz}
    L\msb{3-1100\mr{\mu m}} = \sum c_i \nu L_{\nu}\left(i\right)
\end{equation}

where $L\msb{3-1100\,\mu m}\,L_{\odot}$ is the total IR luminosity in the frequency range 3 to 1100 $\mr{\mu m}$, $c_i$ is the coefficients at $i = 24,\,70,\,100,\,160,\,250\,\mathrm{\mu m}$, and $L_{\nu}\,L_{\odot}\,{\rm Hz}^{-1}$ is the luminosity at the frequency $\nu$ corresponds to a specific wavelength $i$.
For deriving $L_{\nu}\,L_{\odot}\,{\rm Hz}^{-1}$, we calculate it from flux values and the distance to each galaxy.
We referred to \citet{Cortese2012} for obtaining galaxy distances.

They have derived the conversion equation with at least two bands.
Therefore, we could estimate total IR luminosities even if galaxies are lack of a few flux data.
Since several calculations in the following sections require the total IR luminosity among $8 - 1000\,\mr{\mu m}$ ($L\msb{8-1000\,\mu m}$), we recalibrate this luminosity by multiplying the constant value (0.88) in \citet{Takeuchi2005}.

The total IR luminosity calculated here is almost consistent with that of \citet{Ciesla2014}, although there is a little scatters (Figure~\ref{fig:tircomparison}).

\begin{figure}[htbp]
	\centering
	\includegraphics[width=.6\linewidth]{Chapter_4/Figures/Method_TIRcomparison.pdf}
    \caption[The comparison of total IR luminosities]{\label{fig:tircomparison}
    This figure shows the comparison of total IR luminosity at $8-1000\,\mr{\mu m}$ from different methods.
            Here, we show all HRS galaxies which have both luminosities.
            Blue plots show galaxy samples selected in Section~\ref{sec:crossmatching} and~\ref{sec:reducegalaxysamples}, and red ones show other HRS galaxies.
        Although there are some galaxies whose luminosities below $10^9\,L_{\odot}$ have a relatively larger scatters, the difference of luminosities for our samples (blue plots) is smaller than factor 2.
    }
\end{figure}

Hereafter, we use $L_{8-1000\mathrm{\mu m}}$ obtained here to calculate the $q$ parameter and SFR in Section~\ref{subsec:calculatingsfr}.



\section{Fitting to $q$ parameters}\label{subsec:fittingtoq}
Here, we show how to do the fitting for obtaining the spectral index $\gamma$ which is a principal value to estimate SFR using low-frequency radio emissions.
At low frequencies like a few hundred megahertz, the synchrotron emission can be dominant, which emitted from high energy electrons accelerated by the supernova remnant.
In this paper, we assume radio emission has a single power-law on the frequency and adopt the following equation

\begin{equation}\label{eq:q_fitting}
    q_{\nu} = -\gamma\log{\nu} + \beta
\end{equation}
where $q_{\nu}$ a $q$ parameter defined by Equation~\ref{eq:q_def} at $\nu$ [MHz], and $\beta$ is the second fitting parameter.

In this paper, we executed two types of fitting:

\begin{enumerate}
    \item Fitting to only MWA frequencies ($72-231\MHz$)
    \item Fitting to MWA frequencies and $1500\MHz$
\end{enumerate}

The flux data at $1500\MHz$ is obtained from \citet{Boselli2015}, and here we use only high-quality flux data (flag = 1).
These two types of fitting might allow us to judge the correctness of a single power-law assumption.
These fitting results are summarized in Section~\ref{sec:frequencydependency}.



\section{Calculating the SFR}\label{subsec:calculatingsfr}
In this section, we describe how to derive SFR from low-frequency radio emissions.

In this paper, we estimate this SFR, combining Equation~\ref{eq:q_def} with the following equations:

\begin{align}
    \mr{SFR}\msb{IR} &= 3.88 \times 10^{-44}\brp{\frac{L\msb{8-1000\,\mu m}}{\mr{erg}\,s^{-1}}}\label{eq:sfrir}\\
    q_{\nu\,[\mathrm{MHz}]} &= q_{1500\,[\mathrm{MHz}]} + \log{\left(\frac{\nu\,[\mathrm{MHz}]}{1500\,[\mathrm{MHz}]}\right)}^{\gamma}\label{eq:q_nuto1500}
\end{align}
Equation~\ref{eq:sfrir} is calculating SFR using the IR emission from \citet{Murphy2011}, and Equation~\ref{eq:q_nuto1500} shows the difference of the $q$ parameter between a certain wavelength $\nu$ and $1500\MHz$.

Substituting equation~\ref{eq:q_def} and~\ref{eq:q_nuto1500} into equation~\ref{eq:sfrir} yields the following equation to estimate SFR from radio emission at $\nu\MHz$:

\begin{equation}\label{eq:sfrfromradio}
    SFR_{\mathrm{Radio},\,\nu} = 1.46\times10^{-31}\times 10^{q_{1500\mathrm{MHz}}} {\left(\frac{\nu\,[\mathrm{MHz}]}{1500\,[\mathrm{MHz}]}\right)}^{-\alpha_{\mathrm{sync}}} \times L_{\mathrm{Radio},\,\nu}
\end{equation}

In Section~\ref{subsec:sfrfromlowradio}, we show the results of calculating SFR from low-frequency radio using this equation and comparing it with the SFR from other indicator.



\bibliographystyle{mnras}
%\bibliography{example} % if your bibtex file is called example.bib
\bibliography{masterthesis}

	\chapter{Results}
\begin{chapabstract}

Intro intro intro intro.

\end{chapabstract}

Results

	\chapter{Discussions}
\begin{chapabstract}

Intro intro intro intro.

\end{chapabstract}

Discussion!!

    \chapter{Summary}
%\begin{chapabstract}
%
%Intro intro intro intro.
%
%\end{chapabstract}

In this study, we investigate 18 star-forming galaxies in the HRS catalog for their global relation of the low-frequency radio with IR luminosities.
We find a single power-law assumption for the frequency dependence of $\qn$ is valid across MWA frequencies and $1.5\GHz$, and their slope $\gamma$ is consistent with previous studies.
We also investigate the consistency of the radio SFR expected to be an extinction-free indicator.
Our sample shows that the spectral information for each galaxy is needed to estimate its SFR accurately because of a wide variety of $\qn$.
These results suggest that future observation with several radio bands is required at low frequencies for measuring accurate SFRs.

For further understanding of the low-frequency properties in star-forming galaxies, we need more samples from multi-band observation.
The updated GLEAM survey will reveal physical details with an order of magnitude better angular-resolution and sensitivity than the latest survey.


%%%%%%%%%%%%%%%%%%%% REFERENCES %%%%%%%%%%%%%%%%%%

% The best way to enter references is to use BibTeX:

\bibliographystyle{mnras}
%\bibliography{example} % if your bibtex file is called example.bib
\bibliography{masterthesis}

%%%%%%%%%%%%%%%%%%%%%%%%%%%%%%%%%%%%%%%%%%%%%%%%%%

\appendix

\chapter{Galaxy images}\label{chap:galaxyimages}

\section{Matched samples (18 samples used for the analysis)}
\begin{figure}[htbp]
	\centering
	\includegraphics[width=\linewidth]{Figures/AppendixB_galaxyimages.pdf}
    \caption[Galaxy images (6/18 used for the analysis)]{\label{fig:galaxyimages}
        These are the SDSS \citep{Abolfathi2018} i-band images with radio contours from the GLEAM survey at $170-231\MHz$ (solid green lines).
        These galaxies are selected in Section~\ref{sec:crossmatching} and~\ref{sec:reducegalaxysamples}.
        I draw the green contours started from the 3$\sigma$ local noise and increase by a factor of the square root of $2^n$ (``n'' is an integer).
        The green star marker shows the location of a radio source referred to as the GLEAM catalog.
        The size of this marker does not mean any feature of observations.
        The blue circle shows the isophotal optical size at $25\,\mr{mag}\arcs^{-2}$ from \citet{Boselli2010}.
        On the bottom left in each plot, we show the beam size of the GLEAM survey.
    }
\end{figure}

\begin{figure}[htbp]
    \centering
    \includegraphics[width=\linewidth]{Figures/AppendixB_galaxyimages2.pdf}
    \caption[Galaxy images (12/18 used for the analysis)]{\label{fig:galaxyimages2}
        Continuous.
    }
\end{figure}



\section{Matched samples (not used for the analysis)}
\begin{figure}[htbp]
    \centering
    \includegraphics[width=\linewidth]{Figures/AppendixB_galaxyimages_notselected.pdf}
    \caption[Galaxy images (9/15 not used for the analysis)]{\label{fig:galaxyimages_notselected}
        These galaxies are not used for the analysis labeled in Section~\ref{sec:reducegalaxysamples} although they have a radio counterpart in the GLEAM catalog by the cross-matching.
    }
\end{figure}

\begin{figure}[htbp]
    \centering
    \includegraphics[width=\linewidth]{Figures/AppendixB_galaxyimages_notselected2.pdf}
    \caption[Galaxy images (6/15 not used for the analysis)]{\label{fig:galaxyimages_notselected}
        Continuous.
    }
\end{figure}



\section{Suspicious matching samples}
\begin{figure}[htbp]
    \centering
    \includegraphics[width=\linewidth]{Figures/AppendixB_galaxyimages_suspicious.pdf}
    \caption[Galaxy images (suspicious matching)]{\label{fig:galaxyimages_suspicious}
        These galaxies are flagged as a suspicious matching in Section~\ref{sec:crossmatching}
    }
\end{figure}



\section{Not matching samples}
\begin{figure}[htbp]
    \centering
    \includegraphics[width=\linewidth]{Figures/AppendixB_galaxyImages_noMatch.pdf}
    \caption[Galaxy images (no matching)]{\label{fig:galaxyimages_nomatch}
        These galaxies do not have a radio counterpart from the GLEAM catalog in Section~\ref{sec:crossmatching}
    }
\end{figure}



\chapter{Galaxy samples for the analysis}\label{chap:galaxysamples}

\begin{landscape}
    \begin{table}
	\centering
    \caption[The overview of 18 samples used for the analysis]{\label{tab:sampletable}
        This table shows 18 HRS galaxies selected in Section~\ref{sec:crossmatching} and~\ref{sec:reducegalaxysamples}.
        Column (1) NGC (2) R.A. (3) Dec. (5) D25 (6) Dist.\ are from \citet{Cortese2012}, column (4) Type is from \citet{Ciesla2014} (\citealt{Cortese2012} for only HRS 163) and column (7) GLEAM ID refers to GLEAM catalog.
        $\gamma\msb{MWA}$, $\gamma\msb{MWA+1500\,MHz}$ are the result in Section~\ref{sec:fittingtoq}.
        $\mr{SFR}\msb{IR}$ is calculated with the equation from \citet{Murphy2011} and $\mr{SFR\msb{Radio,\,151\MHz}}$ is calculated in Section~\ref{sec:calculatingsfr} with individual parameters.
        ``AGN'' is identified BPT diagram \citep[e.g.][]{Baldwin1981, Kewley2001, Kauffmann2003, Schawinski2007} with emission lines \citep{Boselli2015}.
    \nh~deficient galaxy is identified when the value of \nh-def \citep{Boselli2014} is larger than 0.4.}
    \scalebox{0.8}{

\begin{tabular}{lllllrrlllllll}
\toprule
{} &   NGC &         R.A. &          Dec &      Type &   D25 &  Dist. &        GLEAM ID & $\gamma_{\mr{MWA}}$ & $\gamma_{\mr{MWA+1500MHz}}$ & $\mr{SFR}\msb{IR}$ & $\mr{SFR}\msb{Radio,\,151\MHz}$ &   AGN & HI-def \\
HRS &      &              &              &           &  [arcmin] &  [Mpc] &             &                         &                                 & $\brb{M_{\odot}\,\mr{yr}^{-1}}$ & $\brb{M_{\odot}\,\mr{yr}^{-1}}$          &       &        \\
\midrule
25  &  3437 &  10:52:35.75 &  +22:56:02.9 &        Sc &  2.51 &  18.24 &  J105236+225606 &             0.33+/-0.47 &                   -0.66+/-0.07 &         1.41+/-0.08 &                              0.96+/-0.28 &     - &      - \\
36  &  3504 &  11:03:11.21 &  +27:58:21.0 &       Sab &  2.69 &  21.94 &  J110311+275812 &             -0.42+/-0.1 &                   -0.53+/-0.05 &         4.09+/-0.16 &                              5.15+/-1.35 &     - &   True \\
50  &  3655 &  11:22:54.62 &  +16:35:24.5 &        Sc &  1.55 &  21.43 &  J112254+163522 &             -0.38+/-0.2 &                   -0.63+/-0.04 &         1.91+/-0.07 &                              1.45+/-0.41 &     - &      - \\
77  &  4030 &  12:00:23.64 &  -01:06:00.0 &       Sbc &  4.17 &  20.83 &  J120023-010607 &            -0.57+/-0.07 &                   -0.63+/-0.04 &         4.81+/-0.19 &                              3.55+/-0.94 &     - &      - \\
102 &  4254 &  12:18:49.63 &  +14:24:59.4 &        Sc &  6.15 &  17.00 &  J121850+142515 &             -0.7+/-0.05 &                    -0.7+/-0.03 &         6.47+/-0.25 &                              8.19+/-2.12 &     - &      - \\
114 &  4303 &  12:21:54.90 &  +04:28:25.1 &       Sbc &  6.59 &  17.00 &  J122154+042827 &            -0.57+/-0.05 &                   -0.58+/-0.04 &         6.15+/-0.23 &                              5.49+/-1.43 &     - &      - \\
122 &  4321 &  12:22:54.90 &  +15:49:20.6 &       Sbc &  9.12 &  17.00 &  J122255+154939 &            -0.77+/-0.07 &                              - &         5.45+/-0.22 &                              4.28+/-1.14 &     - &   True \\
144 &  4388 &  12:25:46.82 &  +12:39:43.5 &        Sb &  5.10 &  17.00 &  J122548+123917 &             0.22+/-0.22 &                   -0.73+/-0.08 &         1.66+/-0.06 &                              3.09+/-0.85 &  Seyfert &   True \\
163 &  4438 &  12:27:45.59 &  +13:00:31.8 &        Sb &  8.12 &  17.00 &  J122744+130020 &            -0.81+/-0.35 &                              - &         0.69+/-0.03 &                              2.13+/-0.66 &  Seyfert &   True \\
190 &  4501 &  12:31:59.22 &  +14:25:13.5 &        Sb &  7.23 &  17.00 &  J123159+142503 &            -0.67+/-0.12 &                   -0.72+/-0.05 &         4.37+/-0.18 &                              4.92+/-1.31 &     - &   True \\
201 &  4527 &  12:34:08.50 &  +02:39:13.7 &       Sbc &  5.86 &  17.00 &  J123408+023909 &            -0.54+/-0.06 &                   -0.56+/-0.04 &         4.55+/-0.24 &                              2.71+/-0.71 &     - &      - \\
203 &  4532 &  12:34:19.33 &  +06:28:03.7 &  Im(Im/S) &  2.60 &  17.00 &  J123420+062758 &            -0.55+/-0.13 &                    -0.6+/-0.04 &         0.99+/-0.04 &                              1.52+/-0.43 &     - &      - \\
204 &  4535 &  12:34:20.31 &  +08:11:51.9 &        Sc &  8.33 &  17.00 &  J123418+081157 &             -0.71+/-0.1 &                              - &          2.6+/-0.11 &                              2.69+/-0.75 &     - &      - \\
205 &  4536 &  12:34:27.13 &  +02:11:16.4 &       Sbc &  7.23 &  17.00 &  J123427+021114 &            -0.61+/-0.04 &                   -0.57+/-0.02 &         3.58+/-0.14 &                              2.63+/-0.69 &     - &      - \\
220 &  4579 &  12:37:43.52 &  +11:49:05.5 &        Sb &  6.29 &  17.00 &  J123743+114909 &            -0.42+/-0.18 &                   -0.77+/-0.05 &         1.56+/-0.08 &                              2.32+/-0.65 &  LINER &   True \\
247 &  4654 &  12:43:56.58 &  +13:07:36.0 &       Scd &  4.99 &  17.00 &  J124355+130801 &            -0.27+/-0.29 &                   -0.66+/-0.06 &         2.59+/-0.11 &                              1.65+/-0.49 &     - &      - \\
251 &  4666 &  12:45:08.59 &  -00:27:42.8 &        Sc &  4.57 &  21.61 &  J124508-002747 &            -0.58+/-0.03 &                   -0.58+/-0.02 &          9.2+/-0.33 &                              9.11+/-2.35 &     - &      - \\
306 &  5363 &  13:56:07.21 &  +05:15:17.2 &       pec &  4.07 &  16.23 &  J135607+051516 &            -0.61+/-0.08 &                   -0.51+/-0.04 &         0.27+/-0.03 &                              1.75+/-0.46 &     - &   True \\
\bottomrule
\end{tabular}
}
\end{table}
\end{landscape}





\chapter{Fitting results}\label{chap:fittingresults}
\begin{figure}[htbp]
    \centering
    \includegraphics[width=\linewidth]{Figures/AppendixC_qfitting.pdf}
    \caption[Fitting results for 18 samples]{\label{fig:fittingresults}
        These figures show the fitting result.
        Red points show fluxes at each MWA frequency, and a yellow point shows the flux at $1500\MHz$ \citep{Boselli2015}.
        Blue solid line shows the best fitting line, and the shaded region represents the 95\% confidence interval for the fitting to MWA frequencies.
        Yellow line and shaded area show the fitting result with $1500\MHz$ besides MWA frequencies (For HRS122, 163 and 204, we do not display these because of the lack of high-quality data at $1500\MHz$ data).
        On the bottom right, we show the SDSS-RGB stacking image for each galaxy.
    }
\end{figure}





\end{document}
